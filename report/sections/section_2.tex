\section{Task 2}\label{sec:task2}

The goal of the second task is to implement a software to quantify global differences between ensembles and to identify variance along the sequence. The input of this task are files containing the features of one PED ensemble.

\subsection{Ensembles features}
Relationships between different ensambles will be identified considering the structural features of single ensemble using the features file of the task 1.

At the begining, the software extracts the features for each ensamble (multiple conformations):
\begin{itemize}
\item Radius of gyration.
\item Secondary structure entropy for each position across ensemble conformations using the probability of each model.
\item Median ASA for each position across ensemble conformations using the model ASA.
\item Median RMSD using rotate-translation matrices for each model.
\item Median distance of each pair of equivalent positions. It consider the matrix columns of conformation distance matrices.
\item Standard deviation of the distance of each pair of equivalent positions.
\end{itemize}


\subsection{Global metric}
The global metric evaluates the distance between ensembles pair. 
This metric return the sum of partial distances of different features:
\begin{itemize}
\item The absolute value of the difference between the mean of the radius of gyration for the two ensembles.
\item The chebyshev distance between the entropies of the two ensembles. It finds the distance as the greatest of their differences along any coordinate dimension.
\item The euclidean distance between the median ASA of the two ensembles.
\item The euclidean distance between the median RMSD of the two ensembles.
\item The complementary of the correlation between the median distance of the two ensembles.
\end{itemize}
The value returned by this metric is a weighted distance that takes into account the different nature of the features evaluated.

Using the global metric just described, the software displays the distances between two ensembles through:
\begin{itemize}
\item Heatmap: It calculates the matrix of the distances between the features of the ensambles and then builds and displays the heatmap. In the graph, the distance between ensembles is highlighted with different colors based on the similarity scale on the right. 
\item Dendrogram: Using a linkage matrix and global metric, the distances between pairs of ensembles are evaluated and displayed. The dendrogram combines the most similar ensembles with respect to the calculated features.
\end{itemize}

\begin{figure}[H]
	\begin{minipage}[b]{0.93\textwidth}
		\centering
		\includegraphics[width=\textwidth]{PED00020_heatmap.png}
		\caption{Heatmap of all the model.}
		\label{heatmap}
	\end{minipage}
	\begin{minipage}[b]{0.93\textwidth}
		\centering
		\includegraphics[width=\textwidth]{PED00020_dendrogram.png}
		\caption{Dendrogram of all the model.}
		\label{dendrogram}
	\end{minipage}
\end{figure}

In the previous graphs it is possible to visualize the distance or similarity between the five ensembles within the same PED by comparing the features with the global metric. It can be observed that ensembles 1 and 2 are the most similar (most intense color in the heatmap) and ensemble 4 is the most distant.


\subsection{Local metric}
The local metric evaluates the variability of all the features for each position in the ensemble.
This metric computes the mean of the total values for each residues with respect all the ensamble features:
\begin{itemize}
\item Total Entropy is the standard deviation of the conformations entropy for each position.
\item Total median ASA is the standard deviation of the conformations accessible surface area for each position.
\item Total median RMSD is the standard deviation of the conformations median RMSD for each position.
\item Total standard deviation of distance is the trimmed mean of the conformations median RMSD for each position. 
The use of a trimmed mean helps eliminate the influence of outliers or data points on the tails on the traditional mean.
\end{itemize}

\begin{figure}[H]
	\begin{minipage}[b]{0.97\textwidth}
		\centering
		\includegraphics[width=\textwidth]{PED00020_local.png}
		\caption{Plot of local features.}
		\label{plot}
	\end{minipage}	
\end{figure}
In the previous graph it is possible to visualize the variation all the features of the different ensembles for each residue using the local metric. 

Looking at the graph above and taking into consideration the information extracted from the various comparisons in this project, we can conclude that the structure of the measles virus nucleoprotein is quite disordered in the initial part where the variability is greater. Furthermore, there is a continuous change in the variability of local distances within the structure. There are minimum points where even more defined structures can be identified through pymol images (e.g. alpha-helix).