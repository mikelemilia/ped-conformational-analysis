\section{Task 1}\label{sec:task1}

The goal of the first task is to implement a software to identify conformational relationships within a single ensemble. In this project we use the sructural ensamble PED00020 (measles virus nucleoprotein).
The input of this task is a file containing the PDB structure of one single PED ensemble.
%commentare il fatto che alcuni ensemble hanno zeros aggiunti 

\subsection{Single conformation features}
Relationships within an ensemble will be identified considering the structural features of single conformations.

First, the software extracts the single conformation features and it puts them into a unique dictionary for each model:
\begin{itemize}
\item Radius of gyration of the structure: it is computed using the coordinates of each atom. It computes the barycenter and the distance of each atom from it.
\item Relative accessible surface area (ASA) for each residue: it should be noted that for within a conformation the ASA is the same.
\item Secondary structure for each residue: it computes the angles Phi and Psi and then get SS from them and compare with DSSP. It visualizes the SS regions using the Ramachandran plot
\item Residue distance matrix: it computes the pairwise distances between residues.
\end{itemize}


\subsection{Representative conformations}
In order to extract the representative conformations for each conformation, the software clusters all the models within a single ped using KMedoids methods and a customed metric function. 
We use KMedoids because can be used with arbitrary dissimilarity measures and minimizes a sum of pairwise dissimilarities. 
Furthermore, we use a specific metric function built using the calculated features and different types of distances to measure their distances. The function takes in input two residues and compute their distance that is a sum of the partial features distances.
The partal metrics are: absolute difference (radius of gyration), euclidian distance (ASA), Hamming (SS) and the inversion of correlation (distance matrix). 

Then the software compute the distance among the representative conformations by the same metric function and plot a weighted graph.
The length of the edges is proportional to the distance between the two conformations in the nodes. The label in each node is the ID of the conformations.


% alta variabilità nei centroidi trovati: probabilmente sono molto vicini e l'inizializzazione random fa scegliere uno o l'altro: ci aspettiamo sempre centroidi diversi ma la stessa cardinalità. 

.....significato biologico.... 


\subsection{Pymol image}
???
.....significato biologico.... 